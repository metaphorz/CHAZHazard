\documentclass[11pt]{article}
\usepackage[margin=1in]{geometry}
\usepackage{graphicx}
\usepackage{hyperref}
\usepackage{booktabs}
\usepackage{float}
\usepackage{xcolor}
\usepackage{listings}

\title{CHAZ Florida Hurricane Hazard Map\\[0.5em]\large Interactive Visualization of Tropical Cyclone Wind Hazard Data}
\author{Paul Fishwick \and Claude Code}
\date{February 2026}

\begin{document}

\maketitle

\begin{abstract}
This report documents an interactive web-based visualization tool for exploring tropical cyclone wind hazard data for Florida. The tool displays exceedance intensity data from the Columbia HAZard (CHAZ) model, allowing users to compare historical baseline conditions (1995--2014) with mid-century (2041--2060) and late-century (2081--2100) climate projections. Users can select from three SSP emissions scenarios (SSP245, SSP370, SSP585), six different CMIP6 climate models (or a multi-model mean), and various return periods to explore how hurricane wind hazards may evolve under different future pathways.
\end{abstract}

\section{Introduction}

Tropical cyclones pose significant risks to Florida's coastal communities. Understanding how hurricane wind hazards may change under future climate scenarios is essential for long-term planning, infrastructure design, and risk assessment. This project provides an accessible, interactive tool for exploring CHAZ model output data.

The visualization is available online at: \url{https://metaphorz.github.io/CHAZHazard/}

\section{Data Source}

The underlying data comes from the CHAZ Hazard Maps dataset published on Dryad:

\begin{quote}
Meiler, S., Lee, C.-Y., Sobel, A., \& Camargo, S. (2025). CHAZ Hazard Maps: Global coastal wind hazard maps from the CHAZ tropical cyclone model. Dryad. \url{https://doi.org/10.5061/dryad.qfttdz0vz}
\end{quote}

The dataset provides exceedance intensity values---the wind speeds (in m/s) expected to be exceeded at various return periods (10, 25, 50, 100, 250, and 1000 years)---for coastal locations worldwide. For this visualization, we extracted 2,094 land-based points covering Florida.

\section{The CHAZ Model}

\subsection{Overview}

The Columbia HAZard model (CHAZ) is a statistical-dynamical downscaling model for estimating tropical cyclone hazard. Rather than predicting specific future hurricanes, CHAZ generates thousands of synthetic storms based on climate conditions from global climate models (GCMs), then aggregates statistics to estimate hazard metrics.

\subsection{Model Components}

CHAZ consists of three primary modules:

\begin{enumerate}
    \item \textbf{Genesis Module}: Uses the Tropical Cyclone Genesis Index (TCGI) to estimate where and when storm precursors form based on environmental conditions.

    \item \textbf{Track Module}: Employs a Beta-Advection Model (BAM) that moves storms forward using environmental steering flow from the climate model.

    \item \textbf{Intensity Module}: Evolves storm intensity using an autoregressive model with deterministic forcing (based on potential intensity, wind shear, humidity) and stochastic elements.
\end{enumerate}

\subsection{Synthetic Track Generation}

Understanding how CHAZ creates synthetic hurricane tracks helps interpret the hazard data. Each synthetic storm is generated through a multi-step process that samples meteorological variables from the driving climate model.

\subsubsection{Genesis (Storm Formation)}

CHAZ begins by randomly placing potential storm ``seeds'' across ocean basins. Each seed has a probability of developing into a tropical cyclone based on local environmental conditions:

\begin{itemize}
    \item \textbf{Sea Surface Temperature (SST)}: Warm water ($>$26°C) provides energy for storm development
    \item \textbf{Vertical wind shear}: Low shear allows the storm vortex to remain vertically aligned
    \item \textbf{Mid-level humidity}: Moist air at 500--700 hPa supports deep convection
    \item \textbf{Low-level vorticity}: Pre-existing rotation at 850 hPa helps spin-up
\end{itemize}

Seeds that survive this probabilistic filtering become synthetic storms with assigned genesis locations and times.

\subsubsection{Track Propagation}

Once formed, synthetic storms are advected forward in time (typically 6-hour steps) using a Beta-and-Advection model:

\begin{equation}
\vec{V}_{storm} = \vec{V}_{steering} + \vec{V}_{beta} + \vec{V}_{random}
\end{equation}

where:
\begin{itemize}
    \item $\vec{V}_{steering}$ is the deep-layer mean wind (850--200 hPa average) from the climate model
    \item $\vec{V}_{beta}$ is the beta drift---a systematic poleward and westward motion caused by the Coriolis effect acting on the storm's circulation
    \item $\vec{V}_{random}$ is a stochastic component that captures track variability not explained by the large-scale flow
\end{itemize}

The track integration continues until the storm dissipates (over cold water or land) or exits the model domain.

\subsubsection{Intensity Evolution}

Storm intensity evolves according to the balance between intensification and weakening factors:

\begin{itemize}
    \item \textbf{Potential Intensity (PI)}: The theoretical maximum intensity given local SST and atmospheric thermodynamic profile---storms tend to intensify toward this ceiling
    \item \textbf{Wind shear}: Tilts the vortex and ventilates warm core air, causing weakening
    \item \textbf{Ocean feedback}: Storm-induced upwelling brings cold water to the surface, reducing the energy supply
    \item \textbf{Land interaction}: Storms weaken rapidly over land due to loss of oceanic energy and increased surface friction
\end{itemize}

The intensity model uses an autoregressive formulation where current intensity depends on previous intensity plus forcing terms, with stochastic elements to capture the inherent unpredictability of rapid intensification events.

\subsubsection{Environmental Variables Sampled}

Table~\ref{tab:chaz_vars} summarizes the key variables CHAZ samples from each climate model.

\begin{table}[H]
\centering
\caption{Environmental variables used by CHAZ for synthetic track generation}
\label{tab:chaz_vars}
\begin{tabular}{lll}
\toprule
\textbf{Variable} & \textbf{Source} & \textbf{Role in CHAZ} \\
\midrule
Sea Surface Temperature & Climate model ocean & Genesis probability, intensity fuel \\
Steering winds (850--200 hPa) & Climate model atmosphere & Track direction and speed \\
Vertical wind shear & Climate model atmosphere & Genesis and intensity limiter \\
Mid-level humidity & Climate model atmosphere & Genesis probability \\
Potential Intensity & Thermodynamic calculation & Intensity ceiling \\
Low-level vorticity & Climate model atmosphere & Genesis probability \\
\bottomrule
\end{tabular}
\end{table}

\subsubsection{Ensemble Generation}

To build robust statistics, CHAZ generates large ensembles of synthetic storms:

\begin{itemize}
    \item \textbf{80 ensemble members} per climate model/scenario/period combination
    \item Each member represents a statistically plausible realization of storms consistent with the climate conditions
    \item The stochastic components in genesis, track, and intensity ensure that each ensemble member differs
    \item Return period statistics are computed from the combined distribution of all ensemble members
\end{itemize}

This ensemble approach is why different climate models produce different hazard patterns---each model projects different future SST, shear, and steering flow patterns, which CHAZ translates into different synthetic storm climatologies.

\subsection{Return Period Calculation}

A common question is how CHAZ calculates return periods (e.g., 100-year or 1000-year wind speeds) from a historical period of only 20 years. The key is that CHAZ uses the \textit{climate conditions} from each period to generate thousands of synthetic storms---far more than actually occurred.

The process works as follows:

\begin{enumerate}
    \item \textbf{Climate input}: CHAZ reads environmental conditions (SST, wind shear, humidity, etc.) from the climate model for the time period of interest
    \item \textbf{Synthetic storm generation}: Using these conditions, CHAZ generates approximately 1,600 synthetic years of storms (80 ensemble members $\times$ 20 years per member)
    \item \textbf{Wind speed sampling}: For each grid point, CHAZ records the maximum wind speed from every synthetic storm that passes nearby
    \item \textbf{Extreme value statistics}: These thousands of wind speed samples are fitted to an extreme value distribution
    \item \textbf{Return period extraction}: From the fitted distribution, CHAZ calculates the wind speed expected to be exceeded with probability $1/T$ per year, where $T$ is the return period
\end{enumerate}

For example, a 100-year return period wind speed is the value that has a 1\% annual exceedance probability based on the statistical distribution of all synthetic storms. This approach allows robust estimation of rare events (like 1000-year winds) because the synthetic storm ensemble provides sufficient statistical samples, even when the underlying climate period is only 20 years.

\subsection{Model Scope and Limitations}

CHAZ focuses on large-scale environmental factors that drive tropical cyclone behavior:

\begin{itemize}
    \item Sea Surface Temperature (SST)
    \item Potential Intensity (PI)
    \item Vertical wind shear
    \item Mid-level humidity
    \item Atmospheric steering flow
\end{itemize}

The model does account for landfall-induced weakening---storms lose intensity over land due to the loss of oceanic energy and increased surface friction---but this is handled in a simplified, parameterized way.

\textbf{What CHAZ does not include:}

\begin{itemize}
    \item Detailed surface roughness variations (urban vs.\ rural vs.\ forest)
    \item Terrain effects (hills, valleys, elevation changes)
    \item Building density or urban canopy effects
    \item Specific land cover types
\end{itemize}

The wind speeds in this dataset represent maximum sustained winds at standard measurement height (10m) from synthetic storms, without site-specific adjustments for local surface conditions. For detailed local wind hazard assessment, CHAZ output would typically be post-processed with wind field models (e.g., Holland parametric model), surface roughness correction factors, and terrain adjustments.

Additionally, because CHAZ is a \textit{coastal} wind hazard model, data coverage is naturally denser near coastlines where hurricanes make landfall, with sparser coverage in interior regions.

\subsection{Climate Model Input}

The visualization includes data from six CMIP6 climate models, each run under all three SSP scenarios (SSP245, SSP370, SSP585). Users can select any individual model or view a multi-model mean that averages across all six. Table~\ref{tab:models} describes each available model.

\begin{table}[H]
\centering
\caption{Climate models available in the visualization}
\label{tab:models}
\begin{tabular}{lll}
\toprule
\textbf{Model} & \textbf{Institution} & \textbf{Country} \\
\midrule
CESM2 & National Center for Atmospheric Research & USA \\
CNRM-CM6-1 & Centre National de Recherches M\'{e}t\'{e}orologiques & France \\
EC-Earth3 & EC-Earth Consortium & Europe \\
IPSL-CM6A-LR & Institut Pierre Simon Laplace & France \\
MIROC6 & Multiple Japanese Institutions & Japan \\
UKESM1-0-LL & UK Met Office / NERC & UK \\
\midrule
\textit{Multi-Model Mean} & \textit{Average of all 6 models} & --- \\
\bottomrule
\end{tabular}
\end{table}

\subsection{About the CMIP6 Models}

These models are all part of the Coupled Model Intercomparison Project Phase 6 (CMIP6), which provides the foundation for the IPCC Sixth Assessment Reports. They represent the cutting edge of climate simulation, moving beyond simple physics to include complex Earth System processes like the carbon cycle and vegetation dynamics.

\textbf{CESM2} (Community Earth System Model v2) from the National Center for Atmospheric Research (NCAR), USA, is a highly flexible, community-driven model known for its advanced representation of land-ice (ice sheets) and its CAM6 atmospheric component. It is often used as a benchmark due to its extensive documentation and broad scientific community.

\textbf{CNRM-CM6-1} from CNRM (Centre National de Recherches M\'{e}t\'{e}orologiques), France, uses the ARPEGE-Climat atmospheric model with significant improvements in cloud representation, convection, and surface-atmosphere interactions via the SURFEX system. It is highly efficient and frequently used for High Resolution Model Intercomparison (HighResMIP).

\textbf{EC-Earth3} from a European consortium of 27 research institutes across 10 countries is based on the European Centre for Medium-Range Weather Forecasts (ECMWF) Integrated Forecasting System (IFS). Uniquely, it uses the same core as the world's leading weather prediction model, bridging short-term weather forecasting and long-term climate projection.

\textbf{IPSL-CM6A-LR} from Institut Pierre-Simon Laplace (IPSL), France, is a complete Earth System Model including the LMDz atmospheric model and ORCHIDEE land surface model. ``LR'' indicates Low Resolution, enabling very long multi-century simulations. It excels at simulating the carbon cycle and biosphere-atmosphere interactions.

\textbf{MIROC6} (Model for Interdisciplinary Research on Climate) from JAMSTEC, University of Tokyo, and NIES in Japan focuses on improving internal variability---natural climate swings like El Ni\~{n}o. It features improved stratosphere and shallow convection representation, with particular strength in simulating the Madden-Julian Oscillation (MJO) and tropical rainfall patterns.

\textbf{UKESM1-0-LL} from the Met Office Hadley Centre, UK, is a full-complexity Earth System Model that takes the physical core of HadGEM3 and adds fully interactive chemistry, aerosols, and marine/terrestrial biogeochemistry. It is one of the most comprehensive models in CMIP6, used to explore tipping points and complex climate feedbacks.

Table~\ref{tab:components} compares the core components of each model.

\begin{table}[H]
\centering
\caption{Core components of the CMIP6 climate models}
\label{tab:components}
\begin{tabular}{llll}
\toprule
\textbf{Model} & \textbf{Atmosphere} & \textbf{Land Surface} & \textbf{Ocean} \\
\midrule
CESM2 & CAM6 & CLM5 & POP2 \\
CNRM-CM6-1 & ARPEGE-Climat v6.3 & ISBA-CTRIP & NEMO 3.6 \\
EC-Earth3 & IFS & HTESSEL & NEMO 3.6 \\
IPSL-CM6A-LR & LMDz6 & ORCHIDEE & NEMO 3.6 \\
MIROC6 & CCSR-NIES AGCM & MATSIRO & COCO 4.9 \\
UKESM1-0-LL & HadGEM3-GC3.1 & JULES & NEMO 3.6 \\
\bottomrule
\end{tabular}
\end{table}

\subsection{Why Multiple Models?}

Different climate models make different assumptions about physical processes, leading to a range of projections. By including all six models, users can:

\begin{itemize}
    \item Assess \textbf{model uncertainty} by comparing projections across models
    \item Identify \textbf{robust signals} that appear consistently across models
    \item Use the \textbf{multi-model mean} as a consensus estimate that smooths individual model biases
\end{itemize}

\section{Future Emissions Scenarios (SSP)}

The visualization allows users to explore hurricane hazard under three Shared Socioeconomic Pathways (SSPs), which represent different assumptions about future socioeconomic development and greenhouse gas emissions.

\subsection{Understanding SSP Naming}

SSP scenario names combine two components:

\begin{itemize}
    \item \textbf{First digit (1--5)}: The socioeconomic pathway describing societal development
    \item \textbf{Remaining digits}: The radiative forcing level in W/m$^2$ by 2100
\end{itemize}

Table~\ref{tab:ssp_pathways} describes the five socioeconomic pathways defined by the climate research community.

\begin{table}[H]
\centering
\caption{The five Shared Socioeconomic Pathways (SSP1--SSP5)}
\label{tab:ssp_pathways}
\begin{tabular}{lll}
\toprule
\textbf{Pathway} & \textbf{Name} & \textbf{Description} \\
\midrule
SSP1 & Sustainability & Low challenges, green growth, global cooperation \\
SSP2 & Middle of the Road & Moderate challenges, historical development patterns \\
SSP3 & Regional Rivalry & High challenges, fragmented world, limited cooperation \\
SSP4 & Inequality & Mixed challenges, highly unequal development \\
SSP5 & Fossil-fueled Development & Low challenges, high energy demand from fossil fuels \\
\bottomrule
\end{tabular}
\end{table}

The radiative forcing values (26, 45, 70, 85) represent the additional energy trapped in Earth's climate system by 2100, measured in watts per square meter. Higher forcing leads to more warming.

\subsection{Scenarios in This Visualization}

Table~\ref{tab:ssp} shows how our three scenarios map to the SSP framework.

\begin{table}[H]
\centering
\caption{SSP scenarios available in the visualization}
\label{tab:ssp}
\begin{tabular}{lllll}
\toprule
\textbf{Scenario} & \textbf{SSP} & \textbf{Forcing} & \textbf{CO$_2$ by 2100} & \textbf{Warming} \\
\midrule
SSP245 & SSP2 (Middle of Road) & 4.5 W/m$^2$ & $\sim$600 ppm & $\sim$2.7°C \\
SSP370 & SSP3 (Regional Rivalry) & 7.0 W/m$^2$ & $\sim$850 ppm & $\sim$3.6°C \\
SSP585 & SSP5 (Fossil-fueled) & 8.5 W/m$^2$ & $\sim$1100 ppm & $\sim$4.4°C \\
\bottomrule
\end{tabular}
\end{table}

These three scenarios represent a range from moderate mitigation (SSP245) to high emissions (SSP585). SSP126 (strong mitigation, $\sim$1.5°C) and SSP4 variants were not included in the CHAZ dataset.

By comparing hazard maps across SSP scenarios, users can assess:

\begin{itemize}
    \item How emissions pathway choices affect future hurricane hazard
    \item The range of possible futures for Florida's coastal communities
    \item Whether certain regions show consistent hazard increases regardless of scenario
\end{itemize}

\section{Time Periods and Climate Projections}

The visualization includes three time periods, as shown in Table~\ref{tab:periods}.

\begin{table}[H]
\centering
\caption{Time periods available in the visualization}
\label{tab:periods}
\begin{tabular}{lll}
\toprule
\textbf{Period} & \textbf{Years} & \textbf{Description} \\
\midrule
Historical (base) & 1995--2014 & Baseline climate conditions \\
Mid-Century (fut1) & 2041--2060 & Near-future projection \\
Late-Century (fut2) & 2081--2100 & End-of-century projection \\
\bottomrule
\end{tabular}
\end{table}

\subsection{How Future Projections Are Generated}

For each time period, CHAZ:

\begin{enumerate}
    \item Reads climate conditions (SST, winds, humidity, etc.) from the GCM
    \item Calculates genesis probability using TCGI
    \item Seeds thousands of synthetic storms based on genesis rates
    \item Simulates each storm's track using steering winds
    \item Evolves intensity based on environmental conditions
    \item Aggregates statistics across 80 ensemble members ($\sim$1,600 synthetic years)
\end{enumerate}

Key climate variables that change between periods include:

\begin{itemize}
    \item \textbf{Sea Surface Temperature (SST)}: Warmer oceans provide more energy for storms
    \item \textbf{Potential Intensity (PI)}: Maximum theoretical storm strength increases
    \item \textbf{Vertical Wind Shear}: Affects storm formation and intensification
    \item \textbf{Mid-level Humidity}: Influences moisture availability
    \item \textbf{Atmospheric Circulation}: Alters steering currents and track patterns
\end{itemize}

\section{Visualization Features}

The interactive map includes the following features:

\begin{itemize}
    \item \textbf{Future Scenario Selector}: Choose from three SSP emissions scenarios (SSP245, SSP370, SSP585)
    \item \textbf{Climate Model Selector}: Choose from six individual CMIP6 models or the multi-model mean (CESM2 is the default)
    \item \textbf{Time Period Selector}: Switch between historical, mid-century, and late-century projections
    \item \textbf{Return Period Selector}: Choose from 10, 25, 50, 100, 250, or 1000-year return periods
    \item \textbf{Display Mode Selector}: Choose between Circle (points), Heatmap (interpolated surface), or Contour (isolines) visualization
    \item \textbf{Data Point Tooltips}: Hover over the map to display coordinates, wind speed (m/s, km/h, mph), hurricane category, and all return period values
    \item \textbf{Legend}: Wind speed scale with three unit systems
\end{itemize}

\subsection{Interactive Help Tooltips}

Each dropdown selector in the control panel includes an information icon (\textit{i}). Hovering over any selector label displays a contextual tooltip (dark box appearing to the left) that explains:

\begin{itemize}
    \item \textbf{Future Scenario}: Description of each SSP pathway and its warming implications
    \item \textbf{Climate Model}: Information about CMIP6 models and the multi-model mean
    \item \textbf{Time Period}: Explanation of historical baseline vs.\ future projections
    \item \textbf{Return Period}: Definition of return periods and exceedance probability
    \item \textbf{Display}: Description of each visualization mode (Circle, Heatmap, Contour)
\end{itemize}

These tooltips provide in-context guidance without requiring users to consult external documentation.

\subsection{Display Modes}

The visualization offers three display modes, selectable from the Display dropdown:

\begin{itemize}
    \item \textbf{Circle}: Individual data points displayed as colored circles. This is the default mode and shows the raw grid resolution of the CHAZ model output (Figure~\ref{fig:display_circle}).
    \item \textbf{Heatmap}: A continuous color surface created using Inverse Distance Weighting (IDW) interpolation. This mode provides a smooth visualization of the spatial wind speed gradient across Florida (Figure~\ref{fig:display_heatmap}).
    \item \textbf{Contour}: Isolines connecting points of equal wind speed (in mph). This mode uses the marching squares algorithm with segment stitching to create continuous contour lines. Labels appear on all reasonably large contours (Figure~\ref{fig:display_contour}).
\end{itemize}

All three modes support hover tooltips showing detailed wind speed information for the nearest data point.

\begin{figure}[H]
\centering
\includegraphics[width=\textwidth]{display_circle.png}
\caption{Circle display mode showing individual data points as colored markers. Each point represents a CHAZ model grid cell with wind speed indicated by color.}
\label{fig:display_circle}
\end{figure}

\begin{figure}[H]
\centering
\includegraphics[width=\textwidth]{display_heatmap.png}
\caption{Heatmap display mode using IDW interpolation to create a continuous color surface. The interpolation is bounded to Florida land areas to prevent extension into water.}
\label{fig:display_heatmap}
\end{figure}

\begin{figure}[H]
\centering
\includegraphics[width=\textwidth]{display_contour.png}
\caption{Contour display mode showing isolines of equal wind speed. Labels indicate wind speed in mph. The marching squares algorithm with segment stitching creates smooth, continuous contour lines.}
\label{fig:display_contour}
\end{figure}

\section{Results}

Figure~\ref{fig:historical} shows the 250-year return period wind speeds for the historical baseline (1995--2014). Figure~\ref{fig:future} shows the same metric for the late-century projection (2081--2100).

\begin{figure}[H]
\centering
\includegraphics[width=\textwidth]{fig_historical.png}
\caption{250-year return period wind speeds for Florida under historical climate conditions (1995--2014). Colors indicate wind intensity from blue (lower) to red (higher).}
\label{fig:historical}
\end{figure}

\begin{figure}[H]
\centering
\includegraphics[width=\textwidth]{fig_future.png}
\caption{250-year return period wind speeds for Florida under late-century climate projections (2081--2100, SSP585). Comparison with Figure~\ref{fig:historical} reveals changes in hazard distribution.}
\label{fig:future}
\end{figure}

\subsection{Key Observations}

Comparing the historical and future projections reveals several patterns consistent with climate change projections for tropical cyclones:

\begin{itemize}
    \item Overall wind intensities show increases in many coastal areas
    \item The spatial pattern of highest hazard remains concentrated along the southeast coast and Keys
    \item Some inland areas show modest changes in projected hazard
\end{itemize}

\section{Technical Implementation}

The visualization is built using:

\begin{itemize}
    \item \textbf{Leaflet.js}: Open-source JavaScript library for interactive maps
    \item \textbf{HTML/CSS/JavaScript}: Standard web technologies for broad compatibility
    \item \textbf{GitHub Pages}: Free hosting for static web content
\end{itemize}

Data is embedded directly in the HTML file to avoid CORS restrictions and enable offline use. The approximately 2,094 land points are rendered efficiently using Leaflet's CircleMarker class.

\section{Data Processing}

The original CHAZ data covers global coastal areas with over 1 million points per model. For this visualization:

\begin{enumerate}
    \item Florida points were extracted using a bounding box (lat: 24--31°N, lon: 88--79.5°W)
    \item Ocean points were filtered out using a Florida land polygon
    \item Data for all three SSP scenarios, six climate models, and three time periods was extracted
    \item The multi-model mean was computed by averaging wind speeds across all six models at each grid point for each SSP/period combination
    \item Wind speeds were rounded to one decimal place to reduce file size
    \item All data (3 SSP scenarios $\times$ 7 model options $\times$ 3 time periods $\times$ 2,134 points) was embedded in the HTML
\end{enumerate}

The resulting standalone HTML file is approximately 15 MB and requires no server---users can simply open it in any modern web browser.

\section{Acknowledgments}

This visualization was developed by Paul Fishwick with assistance from Claude Code. The underlying CHAZ model was developed by the Columbia University team: Drs. Suzana J. Camargo, Chia-Ying Lee, Michael K. Tippett, and Adam H. Sobel.

\section{License}

The visualization code is released under the MIT License. The underlying CHAZ hazard data is released under Creative Commons Zero (CC0).

\section{References}

\begin{enumerate}
    \item Lee, C.-Y., Tippett, M.K., Sobel, A.H., and Camargo, S.J. (2018). An environmentally forced tropical cyclone hazard model. \textit{Journal of Advances in Modeling Earth Systems}, 10, 233--241.

    \item Lee, C.-Y., Camargo, S.J., Sobel, A.H., and Tippett, M.K. (2020). Statistical-dynamical downscaling projections of tropical cyclone activity in a warming climate. \textit{Journal of Climate}, 33, 4815--4834.

    \item CHAZ GitHub Repository: \url{https://github.com/cl3225/CHAZ}
\end{enumerate}

\end{document}
